\chapter{Случайные события}

\section{Определение пространства элементарных исходов, примеры. Понятие события (нестрогое), следствие события, невозможное и достоверное события, примеры. Операции над событиями. Сформулировать классическое определение вероятности и доказать его следствия.}

Случайным называют эксперимент, результат которого невозможно точно предсказать заранее.

Множество $\Omega$ всех исходов данного случайного эксперимента называют \textbf{пространством элементарных исходов}, при этом выполняются следующие условия: 

\begin{itemize}
	\item каждый элементарный исход является <<неделимым>>, т. е. он не может быть разбит на более <<мелкие>> исходы;
	
	\item в результате каждого эксперимента обязательно имеет место ровно один из входящих в $\Omega$ элементарных исходов.
\end{itemize}

\textbf{Пример №1.} Бросают монетку. Возможные исходы: выпадение герба или решки. Тогда $\Omega~=~\{$Герб,~Решка$\}$~--~множество~элементарных~исходов. $|\Omega| = 2$.

\textbf{Пример №2.} Бросают игральную кость: $\Omega~=~\{$<<1>>, <<2>>, <<3>>, <<4>>, <<5>>, <<6>>$\}, |\Omega| = 6.$

\textbf{Пример №3.} Из колоды в 36 карт последовательно извлекают 2 карты (без возвращения). Исход можно описать парой $(x_1, x_2)$, где $x_i$ -- номер карты при $i$-ом извлечении. Тогда $\Omega~=~\{(x_i,~x_j):~x_i,~x_j~\in~\{1, ..., 36\},~i~\neq~j\}, |\Omega| = 36 \cdot 35.$ 


\textbf{Нестрогое определение.} Событием будем называть произвольное подмножество множества элементарных исходов $\Omega$.

\textbf{Пример №4.} Бросают игральную кость: $\Omega~=~\{$<<1>>, <<2>>, <<3>>, <<4>>, <<5>>, <<6>>$\}, |\Omega| = 6.$ Можно определить событие $A$ = \{выпавшее число очков равно <<5>> или <<6>>\}, т.е. $A = \{$<<5>>, <<6>>$\}, |A| = 2$. Если в результате эксперимента выпавшее число очков равно <<5>> или <<6>>, то всё событие $A$ целиком <<наступило>>.

Событие $A$ называют \textbf{следствием события} $B$, если из того, что произошло $B$, следует то, что произошло $A$, т. е. $B \subseteq A$.

Любое множество $\Omega$ содержит в себе два подмножества: $\emptyset$ и $\Omega$. События, соответствующие данным множествам, называются \textbf{невозможным и достоверным} соответственно. Эти события называются несобственными событиями. Все остальные события называются собственными.

\textbf{Пример №5.} Из урны, содержащей два красных и три синих шара, извлекают один шар. Возможные события: $A =$~\{извлечённый шар является красным или синим\} -- является достоверным, $B =$~\{извлечён белый шар\} -- невозможным.

\subsection*{Операции над событиями}

События -- множества элементарных исходов. Следовательно, над ними можно выполнять все операции над множествами. При этом вводится следующая терминология.

\begin{enumerate}
	\item Объединение множеств принято называть суммой событий: $A \cup B = A + B$;
	\item Пересечение множеств называют произведением событий: $A \cap B = A \cdot B$;
	\item Дополнение $A$ называют событием, противоположным $A$: $\overline{A} = \Omega \setminus A$. 	
\end{enumerate}

\subsection*{Классическое определение вероятности}

Пусть:

\begin{enumerate}[label=\arabic*)]
	\item $\Omega$ -- пространство исходов некоторого случайного эксперимента, $|\Omega| = N < \infty$;
	\item все элементарные исходы равновозможны;
	\item существует событие $A \subseteq \Omega, |A| = N_A$.
\end{enumerate}

Тогда вероятностью осуществления события $A$ называют число $P(A) = \frac{N_A}{N}.$

\subsection*{Свойства}

\begin{enumerate}
	\item $P(A) \geq 0$.
	
	\textbf{Доказательство.} Т. к. $N_A \geq 0, N > 0$, то $P(A) = \frac{N_A}{N} \geq 0$.
	
	\item $P(\Omega) = 1$.
	
	\textbf{Доказательство.} $P(\Omega) = \frac{N_\Omega}{N} = \frac{N}{N} = 1, |\Omega| = N_\Omega = N$.
	
	\item Если $A, B$ -- несовместные события, то $P(A + B) = P(A) + P(B)$.
	
	\textbf{Доказательство.}  Т. к. $\Omega$ -- конечно, тогда $A, B \subseteq \Omega$ -- конечны. Существует формула: $|A + B| = |A| + |B| - |AB|$. Т. к. $A$, $B$ -- несовместны, то $AB = \emptyset \Longrightarrow N_A + N_B = N_{A + B}$. Таким образом, $P(A + B) = \frac{N_{A + B}}{N} = \frac{N_A + N_B}{N} = \frac{N_A}{N} + \frac{N_B}{N} = P(A) + P(B)$.
\end{enumerate}

\section{Определение пространства элементарных исходов, примеры. Понятие события (нестрогое). Сформулировать геометрическое и статистическое определения вероятности. Достоинства и недостатки этих определений.}

Случайным называют эксперимент, результат которого невозможно точно предсказать заранее.

Множество $\Omega$ всех исходов данного случайного эксперимента называют \textbf{пространством элементарных исходов}, при этом выполняются следующие условия: 

\begin{itemize}
	\item каждый элементарный исход является <<неделимым>>, т. е. он не может быть разбит на более <<мелкие>> исходы;
	
	\item в результате каждого эксперимента обязательно имеет место ровно один из входящих в $\Omega$ элементарных исходов.
\end{itemize}

\textbf{Пример №1.} Бросают монетку. Возможные исходы: выпадение герба или решки. Тогда $\Omega~=~\{$Герб,~Решка$\}$~--~множество~элементарных~исходов. $|\Omega| = 2$.

\textbf{Пример №2.} Бросают игральную кость: $\Omega~=~\{$<<1>>, <<2>>, <<3>>, <<4>>, <<5>>, <<6>>$\}, |\Omega| = 6.$

\textbf{Пример №3.} Из колоды в 36 карт последовательно извлекают 2 карты (без возвращения). Исход можно описать парой $(x_1, x_2)$, где $x_i$ -- номер карты при $i$-ом извлечении. Тогда $\Omega~=~\{(x_i,~x_j):~x_i,~x_j~\in~\{1, ..., 36\},~i~\neq~j\}, |\Omega| = 36 \cdot 35.$ 

\textbf{Нестрогое определение.} Событием будем называть произвольное подмножество множества элементарных исходов $\Omega$.

\textbf{Пример №4.} Бросают игральную кость: $\Omega~=~\{$<<1>>, <<2>>, <<3>>, <<4>>, <<5>>, <<6>>$\}, |\Omega| = 6.$ Можно определить событие $A$ = \{выпавшее число очков равно <<5>> или <<6>>\}, т.е. $A = \{$<<5>>, <<6>>$\}, |A| = 2$. Если в результате эксперимента выпавшее число очков равно <<5>> или <<6>>, то всё событие $A$ целиком <<наступило>>.

\subsection*{Геометрическое определение вероятности}

Пусть:
\begin{enumerate}[label=\arabic*)]
	\item $\Omega \subseteq \mathbb{R}^n$;
	\item $\mu(\Omega) < \infty$, где $\mu$ -- мера множества;
	\item возможность принадлежности некоторого элементарного исхода случайного эксперимента событию $A$ пропорциональна мере этого события и не зависит от формы $A$ и его расположения внутри $\Omega$.
\end{enumerate}

Тогда вероятностью осуществления события $A$ называется число $P(A) = \frac{\mu(A)}{\mu(\Omega)}$.

Геометрическое определение вероятности является обобщением классического определения на случай, когда $|\Omega| = \infty$.

\textbf{Недостаток} геометрического определения заключается в том, что оно не учитывает возможность того, что некоторые области внутри $\Omega$ окажутся более предпочтительными, чем другие.

\subsection*{Статистическое определение вероятности}

Пусть:
\begin{enumerate}[label=\arabic*)]
	\item некоторый случайный эксперимент произведён $n$ раз;
	\item при этом некоторое наблюдаемое в этом эксперименте событие $A$ произошло $n_A$ раз.
\end{enumerate}

Тогда вероятностью осуществления события $A$ называют эмпирический (т. е. найденный экспериментальным путём) предел: $\lim\limits_{n \rightarrow \infty} = \frac{n_A}{n}$.

\textbf{Недостатки} статичестического определения вероятности: 
\begin{enumerate}[label=\arabic*)]
	\item никакой эксперимент не может быть произведён бесконечно много раз;
	\item с точки зрения современной математики статистическое определение является архаизмом, т. к. не даёт достаточно базы для дальнейшего построения теории.
\end{enumerate}

\section{Определение пространства элементарных исходов, примеры. Сформулировать определение сигма--алгебры событий. Доказать простейшие свойства сигма--алгебры. Сформулировать аксиоматическое определение вероятности.}

Случайным называют эксперимент, результат которого невозможно точно предсказать заранее.

Множество $\Omega$ всех исходов данного случайного эксперимента называют \textbf{пространством элементарных исходов}, при этом выполняются следующие условия: 

\begin{itemize}
	\item каждый элементарный исход является <<неделимым>>, т. е. он не может быть разбит на более <<мелкие>> исходы;
	
	\item в результате каждого эксперимента обязательно имеет место ровно один из входящих в $\Omega$ элементарных исходов.
\end{itemize}

\textbf{Пример №1.} Бросают монетку. Возможные исходы: выпадение герба или решки. Тогда $\Omega~=~\{$Герб,~Решка$\}$~--~множество~элементарных~исходов. $|\Omega| = 2$.

\textbf{Пример №2.} Бросают игральную кость: $\Omega~=~\{$<<1>>, <<2>>, <<3>>, <<4>>, <<5>>, <<6>>$\}, |\Omega| = 6.$

\textbf{Пример №3.} Из колоды в 36 карт последовательно извлекают 2 карты (без возвращения). Исход можно описать парой $(x_1, x_2)$, где $x_i$ -- номер карты при $i$-ом извлечении. Тогда $\Omega~=~\{(x_i,~x_j):~x_i,~x_j~\in~\{1, ..., 36\},~i~\neq~j\}, |\Omega| = 36 \cdot 35.$ 

\subsection*{Определение сигма--алгебры событий}

Пусть:
\begin{enumerate}[label=\arabic*)]
	\item $\Omega$ -- пространство элементарных исходов некоторого случайного эксперимента;
	\item $\beta \neq \emptyset$ -- набор подмножеств в множестве $\Omega$.
\end{enumerate}

Тогда $\beta$ называется сигма--алгеброй событий, если выполнены условия:

\begin{enumerate}[label=\arabic*)]
	\item Если $A \in \beta$, то $\overline{A} \in \beta$;
	\item Если $A_1, ..., A_n, ... \in \beta$, то $A_1 + ... + A_n + ... \in \beta$.
\end{enumerate}

\subsection*{Простейшие свойства}

\begin{enumerate}
	\item $\Omega \in \beta$.
	
	\textbf{Доказательство.} $\beta \neq \emptyset$ -- по определению $\Rightarrow$ $\exists A \subseteq \Omega: A \in \beta$ $\Rightarrow$ (аксиома 1) $\overline{A} \in \beta$ $\Rightarrow$ (аксиома 2) $A + \overline{A} \in \beta$. Т. к. $A + \overline{A} = \Omega$ $\Rightarrow$ $\Omega \in \beta$.
	
	\item $\emptyset \in \beta$.
	
	\textbf{Доказательство.} Т. к. $\Omega \in \beta$ $\Rightarrow$ (аксиома 1) $\overline{\Omega} \in \beta, \overline{\Omega} = \emptyset \Rightarrow \emptyset \in \beta$.
	
	\item Если $A_1, ..., A_n, ... \in \beta$, то $A_1 \cdot ... \cdot A_n \cdot ... \in \beta$.
	
	\textbf{Доказательство.} Т. к.  $A_1, ..., A_n, ... \in \beta \Rightarrow$ (аксиома 1) $\overline{A_1}, ...,\overline{A_n}, ... \in \beta \Rightarrow$ (аксиома 2) $\overline{A_1}~+~...~+~\overline{A_n}~+~...~\in~\beta \Rightarrow$ (аксиома 1) $\Rightarrow \overline{\overline{A_1} + ... +\overline{A_n} + ...} \in \beta \Rightarrow$ (по закону Де Моргана) $\overline{\overline{A_1}} \cdot ... \cdot \overline{\overline{A_n}} \cdot~...~\in~\beta \Rightarrow A_1 \cdot ... \cdot A_n \cdot ... \in \beta$.
	
	\item Если $A, B \in \beta$, то $A \setminus B \in \beta$.
	
	\textbf{Доказательство.} Из свойств операций над множествами: $A \setminus B = A \cdot \overline{B}$. $B \in \beta \Rightarrow$ (aксиома 1) $\overline{B} \in \beta \Rightarrow A, \overline{B} \in \beta \Rightarrow$ (свойство 3) $A \cdot \overline{B} \in \beta \Rightarrow A \setminus B \in \beta$.
\end{enumerate}

\subsection*{Аксиоматическое определение вероятности}

Пусть:
\begin{enumerate}[label=\arabic*)]
	\item $\Omega$ -- пространство элементарных исходов некоторого случайного эксперимента;
	\item $\beta$ -- сигма--алгебра на $\Omega$.
\end{enumerate}

Тогда вероятностью (вероятностной мерой) называется функция $P: \beta \rightarrow \mathbb{R}$ обладающая следующими свойствами: 

\begin{enumerate}[label=\arabic*)]
	\item $\forall A \in \beta \Rightarrow P(A) \geq 0$ (аксиома неотрицательности);
	\item $P(\Omega) = 1$ (аксиома нормированности);
	\item если $A_1, ..., A_n,...$ -- попарно несовместные события, то $P(A_1 +...+A_n +...) = P(A_1)+...+P(A_n)+...$ (расширенная аксиома сложения).
\end{enumerate}

\section{Определение пространства элементарных исходов, примеры. Сформулировать определение сигма--алгебры событий. Сформулировать аксиоматическое определение вероятности и доказать простейшие свойства вероятности.}

Случайным называют эксперимент, результат которого невозможно точно предсказать заранее.

Множество $\Omega$ всех исходов данного случайного эксперимента называют \textbf{пространством элементарных исходов}, при этом выполняются следующие условия: 

\begin{itemize}
	\item каждый элементарный исход является <<неделимым>>, т. е. он не может быть разбит на более <<мелкие>> исходы;
	
	\item в результате каждого эксперимента обязательно имеет место ровно один из входящих в $\Omega$ элементарных исходов.
\end{itemize}

\textbf{Пример~№1.} Бросают монетку. Возможные исходы: выпадение герба или решки. Тогда $\Omega~=~\{$Герб,~Решка$\}$~--~множество~элементарных~исходов. $|\Omega| = 2$.

\textbf{Пример~№2.} Бросают игральную кость: $\Omega~=~\{$<<1>>, <<2>>, <<3>>, <<4>>, <<5>>, <<6>>$\}, |\Omega| = 6.$

\textbf{Пример~№3.} Из колоды в 36 карт последовательно извлекают 2 карты (без возвращения). Исход можно описать парой $(x_1, x_2)$, где $x_i$ -- номер карты при $i$-ом извлечении. Тогда $\Omega~=~\{(x_i,~x_j):~x_i,~x_j~\in~\{1, ..., 36\},~i~\neq~j\}, |\Omega| = 36 \cdot 35.$ 

\subsection*{Определение сигма--алгебры событий}

Пусть:
\begin{enumerate}[label=\arabic*)]
	\item $\Omega$ -- пространство элементарных исходов некоторого случайного эксперимента;
	\item $\beta \neq \emptyset$ -- набор подмножеств в множестве $\Omega$.
\end{enumerate}

Тогда $\beta$ называется сигма--алгеброй событий, если выполнены условия:

\begin{enumerate}[label=\arabic*)]
	\item Если $A \in \beta$, то $\overline{A} \in \beta$;
	\item Если $A_1, ..., A_n, ... \in \beta$, то $A_1 + ... + A_n + ... \in \beta$.
\end{enumerate}

\subsection*{Аксиоматическое определение вероятности}

Пусть:
\begin{enumerate}[label=\arabic*)]
	\item $\Omega$ -- пространство элементарных исходов некоторого случайного эксперимента;
	\item $\beta$ -- сигма--алгебра на $\Omega$.
\end{enumerate}

Тогда вероятностью (вероятностной мерой) называется функция $P: \beta \rightarrow \mathbb{R}$ обладающая следующими свойствами: 

\begin{enumerate}[label=\arabic*)]
	\item $\forall A \in \beta \Rightarrow P(A) \geq 0$ (аксиома неотрицательности);
	\item $P(\Omega) = 1$ (аксиома нормированности);
	\item если $A_1, ..., A_n,...$ -- попарно несовместные события, то $P(A_1 +...+A_n +...) = P(A_1)+...+P(A_n)+...$ (расширенная аксиома сложения).
\end{enumerate}

\subsection*{Простейшие свойства}

\begin{enumerate}
	\item $P(\overline{A}) = 1 - P(A)$.
	
	\textbf{Доказательство.} $A + \overline{A} \in \beta$ (аксиома 2 сигма--алгебры), $A + \overline{A} = \Omega$ $\Rightarrow$ (по свойству вероятностей) $P(\Omega) = 1 = P(A + \overline{A}) \Rightarrow$ т. к. $A, \overline{A}$ -- несовместны, то $P(A + \overline{A}) = P(A) + P(\overline{A}) \Rightarrow P(A) + P(\overline{A}) = 1 \Rightarrow P(\overline{A}) = 1 - P(A)$.
	
	\item $P(\emptyset) = 0$.
	
	\textbf{Доказательство.} $P(\emptyset) = P(\overline{\Omega}) \Rightarrow$ (свойство 1)  $P(\emptyset) = P(\overline{\Omega}) = 1 - P(\Omega) = 1 - 1 = 0, P(\Omega) = 1$ -- аксиома нормированности.
	
	\item Если $A \subseteq B$, то $P(A) \leq P(B)$.
	
	\textbf{Доказательство.} $A \subseteq B \Rightarrow B = A + (B \setminus A)$. Тогда $P(B) = P(A + (B \setminus A)) =$ ($A, B \setminus A$ -- несовместны, аксиома 3) $=P(A) + P(B \setminus A) \geq P(A) \Rightarrow P(B) \geq P(A)$.
	
	\item $\forall A \in \beta: 0 \leq P(A) \leq 1$.
	
	\textbf{Доказательство.} $P(A) \geq 0$ -- из аксиомы 1. $\forall A: A \subseteq \Omega \Rightarrow$~(свойство 3)~$P(A) \leq P(\Omega) \Rightarrow$~(свойство 2)~$P(A) \leq 1$.
	
	\item $P(A + B) = P(A) + P(B) - P(AB)$, где $A, B \in \beta$.
	
	\textbf{Доказательство.} $\forall\forall A, B:$ 
	
	(a) $A + B = A + (B \setminus A), A \cdot (B \setminus A) = \emptyset \Rightarrow$~(аксиома 3)~$P(A+B) = P(A) + P(B \setminus A).$  
	
	(б) $B = AB + (B \setminus A), (AB)(B \setminus A) = \emptyset \Rightarrow$~(аксиома 3)~$P(B) = P(AB) + P(B \setminus A) \Rightarrow P(B \setminus A) = P(B) - P(AB)$.
	
	Из (a) и (б): $P(A + B) = P(A) + P(B) - P(AB)$
	
	\item $\forall \forall A_1, ..., A_n: P(A_1 + ... + A_n) = P(A_1) + ... + P(A_n) - P(A_1A_2) - P(A_1A_3) - ... - P(A_{n-1}A_n) +$ $+ P(A_1A_2A_3) + ... + (-1)^{n+1}P(A_1A_2...A_n).$
	
	\textbf{Доказательство.} Является обобщением свойства 5, может быть доказано с использованием метода математической индукции.
\end{enumerate}

\section{Сформулировать определение условной вероятности. Доказать, что при фиксированном событии $B$ условная вероятность $P(A|B)$ обладает всеми свойствами безусловной вероятности.}

Пусть:

\begin{enumerate}[label=\arabic*)]
	\item $A$ и $B$ -- два события, связанные с одним случайным экспериментом;
	\item известно, что в результате эксперимента произошло событие $B$.
\end{enumerate}

Тогда \textbf{условной вероятностью} осуществления события $A$ при условии, что произошло событие $B$, называется число $P(A|B) = \frac{P(AB)}{P(B)}, P(B) \neq 0.$

\textbf{Теорема.} Пусть:
\begin{enumerate}[label=\arabic*)]
	\item зафиксировано событие $B, P(B) \neq 0$;
	\item $P(A|B)$ рассматривается как функция события $A$.
\end{enumerate}

Тогда $P(A|B)$ обладает всеми свойствами безусловной вероятности.

\textbf{Доказательство.}
\begin{enumerate}
	\item Докажем, что условная вероятность $P(A|B)$ удовлетворяет трём аксиомам вероятности: 
	
	(a) $P(A|B) = \frac{P(AB)}{P(B)}, P(AB) \geq 0, P(B) > 0 \Rightarrow P(A|B) \geq 0.$
	
	(б) $P(\Omega | B) = \frac{P(\Omega B)}{B} = \frac{P(B)}{P(B)} = 1.$
	
	(в) $P(A_1 + ... + A_n + ... | B) = \frac{P((A_1 + ... + A_n + ...)B)}{P(B)} = \frac{1}{P(B)} \cdot P(A_1B + A_2B + ... + A_nB + ...) = |A_1,...,A_n,...$~--~попарно несовместны $\Rightarrow A_1B,...A_nB,...$ -- попарно несовместны, тогда используем расширенную аксиому сложения$| = \frac{1}{P(B)} \cdot [P(A_1B) + P(A_2B) + ... + P(A_nB) + ...] = \frac{P(A_1B)}{P(B)} + ... + \frac{P(A_nB)}{P(B)} + ... =$ $=P(A_1|B) + ... + P(A_n|B) + ...$
	
	\item Т.к. свойства безусловной вероятности являются прямыми следствиями из аксиом безусловной вероятности, а условная вероятность этим аксиомам удовлетворяет, то она удовлетворяет свойствам безусловной вероятности.
\end{enumerate}

\section{Сформулировать определение условной вероятности. Доказать теорему (формулу) умножения вероятностей. Привести пример использования этой формулы.}

Пусть:

\begin{enumerate}[label=\arabic*)]
	\item $A$ и $B$ -- два события, связанные с одним случайным экспериментом;
	\item известно, что в результате эксперимента произошло событие $B$.
\end{enumerate}

Тогда \textbf{условной вероятностью} осуществления события $A$ при условии, что произошло событие $B$, называется число $P(A|B) = \frac{P(AB)}{P(B)}, P(B) \neq 0.$

\textbf{Теорема. Формула умножения вероятностей для двух событий.} Пусть: 
\begin{enumerate}[label=\arabic*)]
	\item $A$ и $B$ -- два события, связанные с одним случайным экспериментом;
	\item $P(A) > 0$.
\end{enumerate}
Тогда $P(AB) = P(A)P(B|A).$

\textbf{Доказательство.} Т.к. $P(A) > 0$, то определена условная вероятность $P(B|A) = \frac{P(AB)}{P(A)} \Rightarrow$ $\Rightarrow~P(AB) = P(A)P(B|A).$

\textbf{Теорема. Формула умножения вероятностей для $n$ событий.} Пусть:
\begin{enumerate}[label=\arabic*)]
	\item $A_1,..., A_n$ -- события, связанные с одним случайным экспериментом;
	\item $P(A_1 \cdot ... \cdot A_{n-1}) > 0$.
\end{enumerate}
Тогда $P(A_1 \cdot A_2 \cdot ... \cdot A_n) = P(A_1)P(A_2|A_1)P(A_3|A_1A_2) \cdot ... \cdot P(A_n|A_1 \cdot ... \cdot A_{n-1}).$

\textbf{Доказательство.}

\begin{enumerate}
	\item $k = \overline{1, n - 1}: A_1~\cdot~...~\cdot~A_k\subseteqq~A_1\cdot...\cdot~A_{n-1}$. По 3-му свойству вероятности $P(A_1\cdot...\cdot A_k)~\geq~P(A_1\cdot \cdot ...\cdot A_{n-1})>0$. Следовательно, все условные вероятности, входящие в правую часть доказываемой формулы, определены, и можно задавать условные вероятности по типу $P(A_n|A_1A_2...A_{n-1})$, и, следовательно, можно пользоваться формулой умножения вероятностей для двух событий.
	
	\item Последовательно применим формулу умножения вероятностей для двух событий: 
	
	$P(A_1 \cdot ... \cdot A_{n-1} \cdot A_n) = P(A_1 \cdot ... \cdot A_{n-2} \cdot A_{n-1}) \cdot P(A_n | A_1 \cdot A_{n-1}) = P(A_1 \cdot ... \cdot A_{n-3} \cdot A_{n-2}) \cdot P(A{n-1} | A_1 \cdot \cdot ... \cdot A_{n-2}) \cdot P(A_n | A_1 \cdot ... \cdot A_{n-1}) = P(A_1)P(A_2|A_1)P(A_3|A_1A_2) \cdot ... \cdot P(A_n|A_1 ... A_{n-1}).$
\end{enumerate}

\textbf{Пример.} На семи карточках написаны буквы слова <<ШОКОЛАД>>. Карточки тщательно перемешивают, и по очереди извлекают случайным образом три из них без возвращения первых карточек. Найти вероятность того, что эти три карточки в порядке появления образуют слово <<ШОК>>:

Событие $A = \{$три карточки в порядке появления образуют слово <<ШОК>>$\}$.

Введет следующие события: 
$A_1 = \{$на первой извлеченной карточке написано <<Ш>>$\}$;
$A_2 = \{$на второй извлеченной карточке написано <<O>>$\}$;
$A_3 = \{$на третьей извлеяенной карточке написано <<К>>$\}$.

Тогда $A = A_1 \cdot A_2 \cdot A_3$.

$P(A) = P(A_1A_2A_3) = P(A_1) \cdot P(A_2|A_1) \cdot P(A_3|A_1A_2) = \frac{1}{7} \cdot \frac{2}{6} \cdot \frac{1}{5} = \frac{1}{105}$.

\section{Сформулировать определение пары независимых событий. Доказать критерий независимости двух событий. Сформулировать определение попарно независимых событий и событий, независимых в совокупности. Обосновать связь этих свойств.}

События $A$ и $B$, связанные с некоторым случайным экспериментом и имеющие ненулевую вероятность, называеются \textbf{независимыми}, если $P(AB) = P(A)P(B)$.

События $A_1,...,A_n$ называются \textbf{попарно независимыми}, если $\forall\forall i \neq j, j \in \{1, ..., n\}: P\{A_iA_j\} = P\{A_i\}P\{A_j\}$.

События $A_1,...,A_n$, связанные с некоторым случайным экспериментом, называются \textbf{независимыми в совокупности}, если $\forall k \in {2, ..., n}, \forall\forall i_1<i_2<...<i_k: P\{A_{i_1},..., A_{i_k}\} = P\{A_{i_1}\} \cdot ... \cdot P\{A_{i_k}\}$.

Если $A_1, ..., A_n$ независимые в совокупности, то $A_1, ..., A_n$ -- попарно независимые. Обратное неверно.

\subsection*{Пример Берништейна.} Рассмотрим правильный тетраэдр, на трех гранях которого написаны цифры <<1>>, <<2>>, <<3>> соответственно, а на четвертой написано <<123>>. Тетраэдр подбрасывают и смотрят, что написано на нижней грани. Докажем, что события $A_i = \{$на нижней грани есть $i\}, i \in {1, 2, 3}$ -- попарно независимы, но не являются независимыми в совокупности.

\begin{enumerate}
	\item $P(A_1) = \frac{1}{2} = P(A_2) = P(A_3)$.
	\item $P(A_1A_2) = P(A_1)P(A_2) = P(A_1A_3) = P(A_1)P(A_3) = P(A_2A_3) = P(A_2)P(A_3) = \frac{1}{4} \Rightarrow A_1, A_2, A_3$ -- попарно независимые.
	\item $P(A_1A_2A_3) = \frac{1}{4} \neq P(A_1)P(A_2)P(A_3) \Rightarrow A_1, A_2, A_3$ -- не являются независимыми в совокупности.
\end{enumerate}

\subsection*{Теорема}

\begin{enumerate}
	\item Если $P(B) > 0$, то $A, B$ -- независимые $\Leftrightarrow P(A|B) = P(A)$.
	\item Если $P(A) > 0$, то $A, B$ -- независимые $\Leftrightarrow P(B|A) = P(B)$.
\end{enumerate}

\subsection*{Доказательство}

\textit{Необходимость.} $A, B$ -- независимые события, т.е. $P(AB) = P(A)P(B)$. $P(A|B) = |$по определению условной вероятности$| = \frac{P(AB)}{P(B)} = \frac{P(A)P(B)}{P(B)} = P(A)$.

\textit{Достаточность.} $P(A|B) = P(A).~P(AB) = |$по формуле умножения вероятностей$|= P(B)P(A|B) = P(B)P(A), P(B) > 0 \Rightarrow A, B$ -- независимые.

Доказательство для (2) аналогично.

\section{Сформулировать определение полной группы событий. Доказать теоремы о формуле полной вероятности и о формуле Байеса. Понятия априорной и апостериорной вероятностей.}

Пусть $\Omega$ -- пространство элементарных исходов, связанных с некоторым случайным экспериментом, а $(\Omega, \beta, P)$ -- вероятностное пространство этого случайного эксперимента.

Говорят, что события $H_1, ..., H_n \in \beta$ образуют полную группу событий, если:
\begin{enumerate}[label=\arabic*)]
	\item $P(H_i) > 0, i = \overline{1, n}$;
	\item $H_iH_j = \emptyset, i \neq j$;
	\item $H_1 + ... + H_n = \Omega$.
\end{enumerate}

\subsection*{Теорема. Формула полной вероятности}

Пусть: 
\begin{enumerate}
	\item $H_1, ..., H_n$ -- полная группа событий.
	\item $P(H_i)>0, i = \overline{1,n}$.
	\item $A \in \beta$ -- событие.
\end{enumerate}
Тогда $P(A) = P(A|H_1)P(H_1)+...+P(A|H_n)P(H_n)$.

\textbf{Доказательство.} $P(A) = P(A\Omega) = P(A(H_1 + ... + H_n)) = P(AH_1 + ... + AH_n) = |H_1, ..., Hn$ -- попарно независимые события$| = P(AH_1) + ... + P(AH_n) = P(A|H_1)P(H_1) + ... + P(A|H_n)P(H_n)$.

\subsection*{Теорема. Формула Байеса}

Пусть: 
\begin{enumerate}
	\item Выполнены условия теоремы о формуле полной вероятности.
	\item $A \in \beta: P(A) > 0$.
\end{enumerate}
Тогда $P(H_i|A) = \frac{P(A|H_i)P(H_i)}{P(A|H_1)P(H_1)+...+P(A|H_n)P(H_n)} = \frac{P(A|H_i)P(H_i)}{P(A)}, i = \overline{1, n}$.

\textbf{Доказательство.} $P(H_i|A) = \frac{P(AH_i)}{A} = |$числитель -- теорема умножения тероятностей, знаменатель -- формула полной вероятности$|=\frac{P(A|H_i)P(H_i)}{P(A|H_1)P(H_1)+...+P(A|H_n)P(H_n)}$.

\textbf{Априорная вероятность} -- это вероятность, которая известна или оценивается до проведения случайного эксперимента.

\textbf{Апостериорная вероятность} -- это вероятность события, рассчитанная после того, как был проведён эксперимент. 

\section{Сформулировать определение схемы испытаний Бернулли. Доказать формулу для вычисления вероятности реализации ровно $k$ успехов в серии из $n$ испытаний по схеме Бернулли. Доказать следствия этой формулы}

Схемой испытаний Бернулли называется серия из однотипных экспериментов указанного вида, в которой отдельные испытания независимы, т.е. вероятность реализации успеха в $i$-ом испытании не зависит от исходов первого,
второго,...,$i-1$-го испытаний.

\subsection*{Теорема}

Пусть проводится серия из $n$ испытаний по схеме Бернулли с вероятностью успеха $p$. Тогда $P_n(k)$ есть вероятность того, что в серии из $n$ испытаний произойдёт ровно $k$ успехов: $P_n(k) = C_n^kp^kq^{n-k}$.

\subsection*{Доказательство}

\begin{enumerate}
	\item Результат проведения серии из $n$ экспериментов запишем с использованием
	кортежа $(x_1, ..., x_n)$, где 
	$x_i =
	\begin{cases} 
		1, & \text{если в \( i \)-м испытании имел место успех}, \\
		0, & \text{если в \( i \)-м испытании имела место неудача}.
	\end{cases}$
	
	\item Пусть \( A = \{\text{в серии из \( n \) испытаний произошло ровно \( k \) успехов}\} \).  
	Тогда \( A \) состоит из кортежей, в которых будет ровно \( k \) единиц и \( n - k \) нулей.  
	В событии \( A \) будет столько элементарных исходов, сколькими способами можно расставить \( k \) единиц по \( n \) позициям.  
	Выбрать \( k \) позиций из имеющихся \( n \) можно \( C_n^k \) способами. Вероятность каждого отдельного исхода равна произведению вероятностей каждого отдельного \( x_i \), и тогда общая вероятность исхода будет равна: $p^k q^{n-k}.$
	Все испытания независимы; следовательно, все кортежи из \( A \) равновероятны, и их \( C_n^k \) штук, что означает: $P_n(k) = C_n^k p^k q^{n-k}$.
\end{enumerate}


\subsection*{Следствие (1)}
Вероятность того, что количество успехов в серии из \( n \) испытаний по схеме Бернулли с вероятностью успеха \( p \) будет заключено между \( k_1 \) и \( k_2 \), равна:$P_n(k_1 \leq k \leq k_2) = \sum_{i=k_1}^{k_2} C_n^i p^i q^{n-i}$.

\subsection*{Доказательство}
\begin{enumerate}
	\item Пусть
	$A_i = \{\text{в серии произошло ровно \( i \) успехов}\}, \quad i = k_1, \dots, k_2$, и $P(A_i) = P_n(i) = C_n^i p^i q^{n-i}$.
	\item Тогда $A = A_{k_1} + A_{k_1+1} + \dots + A_{k_2}$, где $P(A) = P(A_{k_1} + \dots + A_{k_2})$.
	
	Так как события \( A_i \) и \( A_j \) несовместны при \( i \neq j \), то $P(A) = P(A_{k_1}) + \dots + P(A_{k_2})$.
	
	Подставим выражение для \( P(A_i) \):
	$P(A) = \sum_{i=k_1}^{k_2} P\{A_i\} = \sum_{i=k_1}^{k_2} C_n^i p^i q^{n-i}$.
	
	Таким образом:
	$P_n(k_1 \leq k \leq k_2) = \sum_{i=k_1}^{k_2} C_n^i p^i q^{n-i}$.
\end{enumerate}

\section*{Следствие (2)}
Вероятность того, что в серии испытаний Бернулли с вероятностью успеха \( p \) (и неудачи \( q = 1 - p \)) произойдёт хотя бы один успех, равна:$P_n(k \geq 1) = 1 - q^n$.

\section*{Доказательство}
Пусть \( A = \{\text{в серии произошёл хотя бы один успех}\} \).  
Тогда противоположное событие:
$\bar{A} = \{\text{в серии не будет ни одного успеха}\}$.

Вероятность события \( A \) равна:$P(A) = 1 - P(\bar{A})$.

Вероятность \( P(\bar{A}) \) соответствует \( P_n(0) \), то есть вероятности того, что в серии из \( n \) испытаний не произойдёт ни одного успеха:$P(\bar{A}) = P_n(0) = C_n^0 p^0 q^{n-0}$.

Подставляя значения, получаем:$P(\bar{A}) = q^n$.

Таким образом:$P(A) = 1 - P(\bar{A}) = 1 - q^n$.
